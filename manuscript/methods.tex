\section{Methods}

\subsection{Data}

Chromosome number data was obtained from the Chromosome Count Database (CCDB, \citet{rice_2015}) for all taxa from Solanaceae family.
The records obtained were first cleaned using CCDBcurator R package \citep{zenilferguson_2017}, to those we aggregated independent data of chromosome numbers and breeding systems recorded and curated by Emma Goldberg and Boris Igi\'c  (citations for original sources can be found in the supplementary information).
In total we obtained 14,000 records of chromosome number and breeding system than were curated semi automatically by CCDBcurator and by hand to score the chromosome number and the breeding system with the most reconciliation.
Furthermore, for the majority of taxa ploidy was assigned according to information from original sources and Kew Royal Botanic Gardens C-value DNA (bennett citation here dataset and the original sources agreggated by Goldberg and Igi\'c.
For taxa without ploidy information but with information about chromosome number we assigned ploidy based on multiplicity of chromosomes within the genus.
For example, \textit{Solanum betaceum} did not include information about ploidy level but its number of chromosomes is 24, since $x=12$ is the base chromosome number of the \textit{Solanum} genus we assigned  \textit{Solanum betaceum} as diploid. 
Species with more than one ploidy level were assigned smallest most frequent ploidy level recorded.
Domesticated taxa (\ie \textit{Solanum lycopersicon}) were removed to avoid since they contained multiple ploidy levels due to domestication and not spontaneous polyploidy.
The remainder of taxa was then matched to largest time-calibrated phylogeny of Solanaceae and this process resulted in 595 taxa used in each of the diversification models.

Binary or three state classification of ploidy and breeding system for the 595 taxa  was done according to Table \autoref{tab:binaryclassifications}
  
\subsection{Models}
We first defined a binary state speciation and extinction model (BiSSE, \citet{maddison_2007})  for polyploidy evolution where taxa was classified as diploid (D=0) and polyploid (P=1) \cref{table:datatable}. In  a Bayesian framework, we obtained posterior probability distributions of speciation rates ($\lambda_D$, $\lambda_P$), extinction rates ($\mu_D$, $\mu_P$) , net diversification rates ($r_D=\lambda_D-\mu_D, r_P=\lambda_P-\mu_P$), and  relative extinction rates ($\nu_D=\frac{\mu_D}{\lambda_D}, \nu_D=\frac{\mu_D}{\lambda_D}$) just as previously explored in \citet{mayrose_2011}. The cladogenetic changes explored in BiSSE were polyploidization rate are represented by parameter $\rho$ and by diploidization parameter rate $\delta$.\newline 

As a second step, we fitted a hidden state speciation and extinction model (HiSSE, \citet{beaulieu_2016}) to evaluate whether the differences in diversification rates were found due to a hidden trait associated to polyploidy. As discussed by \citet{beaulieu_2016} BiSSE-like models suffer from a large type I error because they fail to account as part of the null hypothesis heterogeneous diversification rate changes  that do not depend on the trait of interest. By including a hidden trait in a state speciation and extinction  model,  HiSSE-like models address the heterogeneous background diversification changes while also parsing out the possible signal of diversification due to the trait of interest. Therefore, HiSSE model for polyploidy linked to diversification has four states: diploid and polyploid subdivided by a binary hidden trait with states A and B for which we estimated the posterior probability distributions of speciation rates ($\lambda_{D_A},\lambda_{D_B}, \lambda_{P_A},\lambda_{P_B}$), extinction rates ($\mu_{D_A},\mu_{D_B}, \mu_{P_A},\mu_{P_B}$),  net diversification rates ($r_{D_A},r_{D_B},r_{P_A},r_{P_B}$), and relative extinction($\nu_{D_A},\nu_{D_B}, \nu_{PA},\nu_{P_B}$). The cladogenetic changes assumed by the fitted model are polyploidization rate $\rho$, diploidization rate $\rho$, and we assumed that the changes between every hidden state are symmetrical with rate $\alpha$. \newline

For breeding system both BiSSE and a HiSSE models were fitted independently by coding data as self-compatible (SC=0) or self-incompatible (SI=1) \cref{table:datatable}. The BiSSE model fitted to breeding system data was done with the goal of investigating if the effect of self-incompatibility in diversification was similar to the pattern found by  \citet{goldberg_2012}. However, the HiSSE model on breeding system parses out if the effect of breeding system in diversification is significant  by minimizing type I error as mentioned above. For these models we assumed that self-compatibility is irreversible as discussed by \citet{igic_2013}, so we defined the transition rate $q_{IC}$ as the parameter for changes from self-incompatible to self-compatible state.\newline

Next, we proposed a multivariate speciation and extinction model (MuSSE, \citet{fitzjohn_2012}) to investigate the link amongst diversification, breeding system, and polyploidy simultaneously. The MuSSE model is defined using three states self-compatible diploids (SC-D=0), polyploids that are always self-compatible (ref for this??) (P=1), and self-incompatible diploids (SI-D=2, see table \cref{table:datatable}) and contains ten parameters, six defining diversification ($\lambda_{SD}, \mu_{SD},\lambda_{P},\mu_{P}, \lambda_{ID},\mu_{ID}$) and the other four key cladogenetic changes that are: polyploidization of self-compatible diploids $\rho_{SD}$, diploidization $\delta$, polyploidization of self-incompatible diploids $\rho_{ID}$, and self-incompatible to self-compatible rate $q_{IC}$.\newline

Since the null hypothesis of the MuSSE model is that the diversification is equal and constant for all three states defined, it is possible that MuSSE can also suffer from large type I errors. In order to account for heterogeneity of the diversification rates and parse the signal in diversification coming from breeding system and polyploidy we extended MuSSE model to account for a hidden state, as \citet{beaulieu_2016} did in the bivariate case. The model we proposed is a multivariate and hidden states speciation and extinction stochastic process (MuHiSSE) that allowed us to account simultaneously for the diversification rates linked to breeding system and polyploidy but also the presence of some more heterogeneity in the process.  This model is analogous to the GeoHiSSE model proposed by Caetano et al. (2018) in a biogeographical context. The full MuHiSSE model has 26 parameters, however, our goal was to look for diversification rate differences so we fitted a simplified version of 17 parameters by fixing the rates amongst hidden states to be equal with parameter $\alpha$ and the transition rates amongst breeding system and polyploidy as defined in the MuSSE model  ($\rho_{SD}$,  $\delta$,  $\rho_{ID}$,  $q_{IC}$) despite the hidden state. Using MuHiSSE we estimated twelve speciation and extinction rates ($\lambda_{SD_A}, \mu_{SD_A},\lambda_{P_A},\mu_{P_A}, \lambda_{ID_A},\mu_{ID_B},\lambda_{SD_B}, \mu_{SD_B},\lambda_{P_B},\mu_{P_B}, \lambda_{ID_B},\mu_{ID_B}$), and the net diversification and relative extinction rates associated with them.\newline

All the models were performed using RevBayes \citep{hoehna_2016} software that performs Bayesian inference via MCMC using Metropolis-Hastings algorithm. A correction for sampling bias was done in all models by assuming that Solanaceae family has approximately 3,000 species ($s=595/3000$) as the Solanaceae Source project indicates \citep{source_2011}.  In all models, speciation and extinction parameters used log-normal prior distributions that with mean the expected net diversification rate $ ( \frac{(number of taxa)/2}{root age})$ and standard deviation $0.5$. Prior distributions for parameters defining cladogenetic changes were gamma distributed with parameters $k=0.5$ and $\theta=1$. MCMC was performed for 96 hours in the cluster at Minnesota Supercomputing Instute which allowed for 5,000 generation of burn-in and a minimum of 200,000 generations of MCMC for each of the 6 models. Convergence of the MCMC was tested using R package coda (ref) and software Tracer (ref) to assess convergence and mixing (see supplementary information).


\begin{table}
\begin{tabular}{@{}llccc@{}} \toprule
\multicolumn{4}{r}{Models} \\ \cmidrule(r){3-5}
Type & Total of & BiSSE/HiSSE & BiSSE/HiSSE & MuSSE/MuHiSSE\\ 
 &Taxa &  Polyploidy & Breeding System &  \\ \midrule
Diploid Self Compatible & 152 & 0 &  0 & 0 \\
Diploid Self Incompatible& 97 & 0  & 1 & 2\\
Diploid with unknown breeding system & 219 & 0 & (0,1) & (0,2) \\
Polyploid & 81 & 1& 0 & 1 \\
Unknow ploidy and self compatible& 34 & (0,1)& 0 & (0,1) \\ 
Unknown ploidy and self incompatible & 12 & 0 & 1 & 2 \\ \bottomrule
\end{tabular}
\caption{Binary and three state classifications for 595 taxa with ploidy and/or breeding system data. The number of taxa in the t he sample was maximize by including tips with only ploidy or only breeding system and assigned them as uncertain in the unknown character.}
\label{table:datatable}
\end{table}
