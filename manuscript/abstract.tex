\section{Abstract}

If particular traits consistently affect rates of speciation and extinction, broad macroevolutionary patterns can be understood as consequences of selection at high levels of the biological hierarchy.
Identifying traits associated with diversification rate differences is complicated by the wide variety of characters under consideration and the statistical challenges of testing for associations from comparative phylogenetic data. %B: complicated by <- tricky, however, because of ... to consider; was %Identifying traits associated with diversification rate differences is tricky, however, because of the wide variety of traits to consider and the statistical challenges of testing for associations from comparative phylogenetic data.
Ploidy (diploid vs.\ polyploid states) and breeding system (self-incompatible vs.\ self-compatible states) have been repeatedly suggested as possibly drivers of differential diversification.
We investigate the connections of these traits, including their interaction, to speciation and extinction rates in Solanaceae.
We show that the effect of ploidy on diversification can be largely explained by its correlation with breeding system and that additional unknown factors, alongside breeding system, determine diversification rates.
%B: was: We find that the effect of ploidy can largely be explained by its correlation with breeding system, and that additional unknown factors work with breeding system to determine diversification rates.
These results are largely robust to allowing for diploidization. %B: left along, but I prefer: "These results are largely robust to inclusion of diploidization."
Finally, we find that the direct pathway, in which breakdowns of self-incompatibility arise as byproducts of polyploidization, rather than a step-wise pathway with breakdowns followed by polyploidization, is disproportionately important in the evolutionary history of the nightshade family.
%B: was: "from self-incompatible diploids lower rate of diversification associated with self-compatible diploids makes it less likely for that state to lie along the evolutionary pathway from self-incompatible diploids to polyploids."