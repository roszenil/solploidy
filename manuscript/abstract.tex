\section{Abstract}

% E: Here's the previous abstract.  The new one below is aligned more with the current Introduction and Results.
%
% The effect of polyploidy in diversification remains a contentious issue. On the one hand,  recent studies  that found that polyploids have slower speciation rates and higher extinction rates than diploids left scientist wondering if polyploidy is truly an evolutionary dead-end. On the other hand, botanist have found strong molecular support of multiple polyploidy events at the root of highly diverse clades which challenges the evolutionary dead-end conclusions reached by modeling approaches. We re-investigate the role of polyploidy in speciation and extinction from a new modeling perspective considering that patterns found in diversification models can be misleading and incorrectly attributed to polyploidy when other observed and unobserved plant traits are responsible of shaping diversification.  Using  statistically robust comparative phylogenetic approaches, we show that it is possible to  detect whether  the contribution of polyploidy to speciation and extinction is significant  under the presence of  other potential traits also affect diversification. We use the phylogeny, polyploidy, and breeding system data of 595 Solanaceae species to understand the contribution of polyploidy to diversification. We ask if Solanaceae polyploids are evolutionary dead-ends, and whether breeding system or some other unobserved traits are responsible of the patterns of diversification observed in the phylogeny.

%B: I think it's fine.
% E:  Help making this less bland?  Except for the first sentence, which may be too grandiose?
If particular traits consistently affect rates of speciation and extinction, broad macroevolutionary patterns can be understood as consequences of selection at high levels of the biological hierarchy.
Identifying traits associated with diversification rate differences is tricky, however, because of the the possibly large selection of traits under consideration and the resulting statistical challenge of testing for associations from comparative phylogenetic data.
%B: grammatical error in "traits ... are whether". Rephrased.
%Two traits that have been repeatedly suggested as drivers of differential diversification are whether a lineage is diploid or polyploid, and whether it is self-incompatible or self-compatible.
Ploidy (diploid vs. polyploid states) and breeding system (self-incompatible vs self-compatible states) have been repeatedly suggested as possibly drivers of differential diversification.
We investigate the role of these traits, including their interaction, on speciation and extinction rates in Solanaceae.
We find that the effect of ploidy can largely be explained by its correlation with breeding system, and that additional unknown factors work with breeding system to determine diversification rates.
These results are largely robust to assumptions about whether diploidization occurs.
Finally, we show that allowing for state-dependent diversification affects conclusions about the relative contribution of different evolutionary pathways to self-compatible polyploids.
