\section{Summary}

\begin{itemize}
\item If particular traits consistently affect rates of speciation and extinction, broad macroevolutionary patterns can be interpreted as consequences of selection at high levels of the biological hierarchy.
Identifying traits associated with diversification rates is difficult because of the wide variety of characters under consideration and the statistical challenges of testing for associations from comparative phylogenetic data.
Ploidy (diploid \vs polyploid states) and breeding system (self-incompatible \vs self-compatible states) are both thought to be drivers of differential diversification in angiosperms. %B: rm: have been repeatedly suggested as possible

\item  We investigate the effect of ploidy and breeding system, including their interaction, on speciation and extinction rates in Solanaceae by fitting and comparing twenty-nine diversification models. %B: rm: connection ...  including interaction to speciation

\item We show that the effect of ploidy on diversification can be largely explained by its correlation with breeding system and additional unobserved factors changing diversification rates. %B: Why not just directly state here that the effect seems to be primarily explained by breeding system? (unaltered)
These results are largely robust to allowing for diploidization.  %B: this sentence was apparently accidentally dropped?
We also find that the most common evolutionary pathway to polyploidy in Solanaceae occurs via direct breakdown of self-incompatibility by whole-genome duplication, rather than indirectly via breakdown followed by polyploidization.

\item  Lineage diversification in Solanaceae appears to be the result of a complex interaction, among ploidy, breeding system, and unobserved factors. 
We show that modeling complex trait interactions is key to our understanding of diversification.        %B was: Our models of trait interactions alongside unobserved factors may enhance our understanding of diversification. 
\end{itemize}