\section{Abstract}

If particular traits consistently affect rates of speciation and extinction, broad macroevolutionary patterns can be understood as consequences of selection at high levels of the biological hierarchy.
Identifying traits associated with diversification rate differences is complicated by the wide variety of characters under consideration and the statistical challenges of testing for associations from comparative phylogenetic data.
Ploidy (diploid \vs polyploid states) and breeding system (self-incompatible \vs self-compatible states) have been repeatedly suggested as possible drivers of differential diversification.
We investigate the connections of these traits, including their interaction, to speciation and extinction rates in Solanaceae.
We show that the effect of ploidy on diversification can be largely explained by its correlation with breeding system and that additional unknown factors, alongside breeding system, determine diversification rates.
These results are largely robust to allowing for diploidization.
Finally, we find that the evolutionary pathway to polyploidy in Solanaceae is most often via the direct breakdown of self-incompatibility by whole genome duplication, rather than indirectly via breakdown followed by polyploidization.
