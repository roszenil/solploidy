\section{Abstract}

The effect of polyploidy in diversification remains a contentious issue. On the one hand,  recent studies  that found that polyploids have slower speciation rates and higher extinction rates than diploids left scientist wondering if polyploidy is truly an evolutionary dead-end. On the other hand, botanist have found strong molecular support of multiple polyploidy events at the root of highly diverse clades which challenges the evolutionary dead-end conclusions reached by modeling approaches. We re-investigate the role of polyploidy in speciation and extinction from a new modeling perspective considering that patterns found in diversification models can be misleading and incorrectly attributed to polyploidy when other observed and unobserved plant traits are responsible of shaping diversification.  Using  statistically robust comparative phylogenetic approaches, we show that it is possible to  detect whether  the contribution of polyploidy to speciation and extinction is significant  under the presence of  other potential traits also affect diversification. We use the phylogeny, polyploidy, and breeding system data of 595 Solanaceae species to understand the contribution of polyploidy to diversification. We ask if Solanaceae polyploids are evolutionary dead-ends, and whether breeding system or some other unobserved traits are responsible of the patterns of diversification observed in the phylogeny.