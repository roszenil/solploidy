\section{Summary}

% Right now at exactly 200 words!
\begin{itemize}
\item If particular traits consistently affect rates of speciation and extinction, broad macroevolutionary patterns can be interpreted as consequences of selection at high levels of the biological hierarchy.
Identifying traits associated with diversification rates is difficult because of the wide variety of characters under consideration and the statistical challenges of testing for associations from comparative phylogenetic data.
Ploidy (diploid \vs polyploid states) and breeding system (self-incompatible \vs self-compatible states) are both thought to be drivers of differential diversification in angiosperms. %B: rm: have been repeatedly suggested as possible

\item  We investigate the effect of ploidy and breeding system, including their interaction, and unobserved traits, on speciation and extinction rates in Solanaceae by fitting and comparing twenty-nine diversification models. %B: add unobserved traits rm: connection ...  including interaction to speciation

\item We show that the diversification patterns in Solanaceae can largely explained by changes in breeding system and additional unobserved factors linked to speciation and rates, even under the presence of ploidy information. %B: Why not just directly state here that the effect seems to be primarily explained by breeding system? (unaltered)
%These results are largely robust to allowing for diploidization.  %B: I now dropped it again to squeeze under 200 words. %B: this sentence was apparently accidentally dropped? % I removed the diploidization because the story was complicated so I preferred not to add it in the abstract
%R- Is the above sentence better now?

We also find that the most common evolutionary pathway to polyploidy in Solanaceae occurs via direct breakdown of self-incompatibility by whole-genome duplication, rather than indirectly via breakdown followed by polyploidization.

\item Comparing multiple stochastic diversification models that include complex trait interactions alongside hidden states enhances our understanding of the macroevolutionary patterns in plant phylogenies.  %B was: We show that modeling complex trait interactions is key to our understanding of diversification.       
\end{itemize}