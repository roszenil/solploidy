\section{Abstract}

If particular traits consistently affect rates of speciation and extinction, broad macroevolutionary patterns can be understood as consequences of selection at high levels of the biological hierarchy.
Identifying traits associated with diversification rate differences is tricky, however, because of the wide variety of traits to consider and the statistical challenges of testing for associations from comparative phylogenetic data.
Ploidy (diploid vs.\ polyploid states) and breeding system (self-incompatible vs.\ self-compatible states) have been repeatedly suggested as possibly drivers of differential diversification.
We investigate the roles of these traits, including their interaction, on speciation and extinction rates in Solanaceae.
We find that the effect of ploidy can largely be explained by its correlation with breeding system, and that additional unknown factors work with breeding system to determine diversification rates.
These results are largely robust to allowing for diploidization.
Finally, we show that the lower rate of diversification for self-compatible diploids makes it less likely for that state to lie along the evolutionary pathway from self-incompatible diploids to polyploids.
