\section{Discussion}


% - value of approach
The present work shows the importance of considering the trait linked diversification patterns under a multivariate approach.
Species are created and go extinct based un multiple and often highly correlated phenotypes, understanding the speciation and extinction processes requires understanding of the evolutionary consequences that those trait correlations produce in organisms
In the present work we show how considering both polyploidy and breeding system can disentangle the importance (or the lack of) of polyploidy when confronted with the evidence brought by breeding system.\newline

% - diversification  (frame as no-delta first)
% - what happens if PD only, Hidden states
Using the most complete dataset for polyploidy in a phylogenetic tree in Solanaceae, we were able to replicate the results found by \citet{mayrose_2011}, polyploids have a slower net diversification compare to diploids
Furthermore,  we also found polyploids had a high probability of having negative diversification which implies that polyploids can become a macroevolutionary dead-end, a result that was also found in the two large angiosperm diversification studies  \citet{mayrose_2011} and \citet{mayrose_2015}
However, we expanded this study to accomodate  background heterogeneity in the diversification process
When adding heterogeneity we found that it was more likely that an unobserved trait linked to diploid state was the one leading the net diversification patterns, and that there were some ``second-class" diploids that were not different from polyploids in diversification terms (Figure 2A)
This result lead us to our central question: \textit{what is that other trait linked diploids that makes them different in the diversification process? }.\newline

%     main finding about our traits
In Solanaceae, our immediate intuition was to look into breeding system
Previous studies shown that self-incompatible Solanaceae species have also higher rates of diversification compared to their self-compatible counterparts\citep{goldberg_2012}
Self-incompatible species are diploid in our sample and also expected to be diploid due to ..
(citation?)
By considering both polyploidy and breeding system simultaneously for every species in our sample, it was possible to disentangle why some diploids were quantitatively different than polyploids
In the three-state diversification model ID/CD/CP, we found that self-incompatible diploids have faster and positive rates than self-compatible diploids and polyploids, and that the difference between the  rates of net diversification of self-compatible diploids is not as large (Figures 2C) as first found by binary trait diversification models (Figures 2A)
This result is important, since it aligns with the net diversification results of the  D/P+A/B model, where a ``hidden-trait" seem to be dictating the diversification pattern
By adding breeding system, we were able to hint at which that hidden-trait possibly is
Therefore, we consider that finding a heterogenous result in the hidden trait approaches should be be treated as evidence of a second trait that is necessary to consider
Pursuing knowledge of  such trait can result on a clearer picture of the importance of trait linked diversification patterns, but also on a better reconstruction on past events in phylogenies.


%     more generally, not okay to approximate musse with bisse when states are correlated (cf Pyron)

% alpha is always low
%     new supp fig showing how A & B are inferred on the tree?  at least for ID/CD/CP+A/B model
%     maybe we can hypothesize factors that underlie the hidden state (geography?)

% pathways: compare/contrast with robertson2011
%     \rho_I > \rho_C in all!

% where WGD is in Solanaceae--motivation and delta rate analyses
% diploidization: comment
%     effect on inferred diversification

% generality of approach
%
% issues
%     meaning of diploidization parameter
%     irreversibility SI -> SC assumption