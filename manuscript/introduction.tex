\section{Introduction}

%NOTE: please separate each line with a line break!

%B: the key alteration below is an attempt at easing the transitions of the complicated existing logical structure, which intermixes advances with setbacks and problems.
%B: the "Here, we" part should maybe come at the bottom of introduction, and it's a little early for here, in the first paragraph.
Species accumulate within the tree of life at different rates.
A possible explanation for this phenomenon is that species possess various traits or character states that may differentially affect rates of diversification. %B: new.
Dramatic increases in available phylogenetic and phenotypic data %citation
as well as methodological advances %citation
have greatly accelerated the search for such traits that influence diversification traits. Nevertheless, identifying focal traits associated with rates of speciation and extinction remains as a difficult statistical problem (see \citep{maddison_2015, rabosky_2015, moore_2016, fitzjohn_2009, goldberg_2012, beaulieu_2016, rabosky_2017}) . %B: the following is now orphaned, because it's inclusion merits addition of non-trait-focused methods; "potentially interesting biological processes"
In part, this is a difficult problem because speciation and extinction rarely depend on a single trait; the context  in which traits occurs can lead to complex interactions resulting in a heterogenous speciation and extinction rates \citep{beaulieu_2016, caetano_2018, herrera_2018}. Consequently, examining the association of only one trait when multiple complex interactions are changing diversification patterns can be misleading.

In flowering plants, both ploidy and breeding system are well-studied traits that have been linked to diversification but may also affect each other \citep{stebbins1950}. 
%B: IMO: ploidy = trait, di-, tri-, tetra-, hexa-, etc. (including "poly-") = trait states; the uncertainty we were discussing is whether "tri-" < "poly-" or "tri-" = "poly-"
%(sorry about the confusion: I accidentally mis-wrote in the commented-out banter with %E.)
% was [B&E banter] and % R- I think the trait is polyploidy and the state is ploidy (as 2, or +2). I will go with everyone else in the literature and use polyploidy
Whole genome duplication (WGD) is a remarkably common mutation in plants \citep{husband_2013, zenilferguson_2017}, resulting in many plant species with a variety of ploidy levels. The prevalence of ploidy variation has been broadly considered as a salient feature of flowering plants for nearly a century \citep{stebbins1938}. 

WGDs have the potential to affect many phentoypes and impact a variety of evolutionary \citep{ramsey_2002} and ecological processes \citep{sessa_2019}.
Polyploids have long been thought to have lower rates of net diversification rate than diploids \citep{mayrose_2011, mayrose_2015}. 
Recent studies, however, find common and numerous paleo-polyploidizations, including some preceding the emergence of highly diverse plant clades \citep{soltis_2014, landis_2018}, suggesting that WGDS have played an important macroevolutionary role driving innovation and diversification in plants.
%B: explanation for this edit, rm: "Using the genomic evidence  from the 1KP project" : this is the data, not the approach; Ks and tree reconciliation are particularly troubling and weak modes of inference. I am okay with having this point to what others have found, but I would prefer to avoid endorsing what I see as a deeply problematic set of methods.
% R- I agree with your point. Tiley, Barker, and Burleigh just published last week a paper about Ks plots and the mixed normal model adjustments to it that "detect" WGDs. They found that in many cases those models lead to overestimating the number of WGDs plus the other issue is using the presence of WGD and then doing something like BAMM that quantitatively does not link via parameters is problematic.
Evidence of paleo-polyploidizations within the genomes of diploid plants also revels the pervasiveness of diploidization, the return of a polyploid to a diploid state throughout the angiosperm phylogeny \citep{soltis_2015diploidization, dodsworth_2015}. Testing for the presence of diploidization in a diversification context is important because the presence of a positive net diversification rate for diploids can be  the result of a polyploid lineage that has been diploidized. Under this scenario, polyploidy should no longer be considered to slow down diversification, but instead, polyploidy would be the initiator of the diversification process. 
%R- This last phrase is from Itay and I rather like it.

Breeding system shifts---changes in the collection of physiological and morphological traits that determine the likelihood that any two gametes unite---are remarkably common and affect the distribution and amount of genetic variation in populations \citep{stebbins1974, barrett2013}.
In particular, self-incompatibility systems cause a plant to reject its own pollen, and their loss, yielding self-compatibility, is one of the most replicated transitions in flowering plant evolution. % refs: stebbins, igic
% Gordon comments here: Really? I find this hard to believe.There are a number of transitions among general selfing and outcrossing mating systems, but I think of SI as only occurring in a limiteed number of angiosperm families. Assigned to Boris
Previous phylogenetic analyses have reported higher rates of diversification for self-incompatible than for self-compatible lineages \citep{goldberg_2010, devos2014}. Faster net diversification of self-incompatible state was also found in the family  \textit{Onagraceae} \citep{freyman_2017} , where authors assumed the possibility of hidden states potentially driving diversification patterns. However, studying breeding system with other correlated traits and unobserved states driving diversification patterns simultaneously has not yet being explored. % Gordon comments here: First, it's a little confusing if you are talking about more SC or SI species more generally, or if you are talking specifically about genetic SI systems. I don't think the de Vos paper is dealing with SI systems. If your'e talking more generally about breeding system evolution there's a vast literature here, and I think  this is oversimplifying. For example, there are a lot of papers trying to link dioecy/hermaphroditism with diversification. (Vamosi papers on dioecy and diversification)
% R- Boris what do you think is the best way to proceed here? Narrow it down to SI systems (deleting de Vos paper) or expanding it to talk about breeding systems in general and then saying that we will only focus on SI systems because of Solanaceae

Ploidy might also affect the propensity for self-fertilization \citep{stebbins1950}. 
The evidence for correlated shifts in mating system, from outcrossing to selfing, after polyploidization appears limited and sometimes contradictory \citep{barringer2007, barrett2008, husband2008}.
In other cases, however, polyploidy is not only suspected to correlate with breeding system but causes the transition from self-incompatibility (SI) to self-compatibility(SC) \citep{stout1942, lewis1947}.
Doubled number of alleles in pollen is thought to effect disruption of the genetic mechanisms in gametophytic SI systems, which prevent self-fertilization \citep{entani1999, tsukamoto2005, kubo2010}. 
This creates a correlation between polyploidy and SC by precluding the existence of SI polyploids.
In clades with these systems, it is thus natural to consider the simultaneous macroevolution of polyploidy and breeding system.
% Gordon comments: Here again is confused about breeding systems in general vs. SI genetic systems

Given that changes in ploidy and breeding systems may be causally related and have profound affects on the fate of lineages, it seems particularly profitable to examine possible interactions in their macroevolutionary effects.
This includes their joint influence on lineage diversification, and also potential patterns in the order of their transitions.
For example, it is possible to ask whether losses of  SI more commonly occurred tied to polyploidization, or without a ploidy shift. Or whether polyploids have arised more commonly from self-incompatible than from self-compatible diploids/
\Citet{robertson_2011} found that the pathway from self-incompatible diploids to self-compatible polyploids is dominated by loss of self-incompatibility followed later by polyploidization over long timescales, but proceeds in one step via polyploidization of SI species over short timescales.
Greatly improved phylogeny and methods that allow for diversification rate differences allow far more powerful approach to such problems. % E: Technically, we allow for div effects in the rate estimates, but not in the pathways.  If SC-D have so much extinction that the step to SC-P never happens, the old pathway methods won't see that.  I'll think about if this is easy to fix.
% And, are diversification rates of polyploid SC lineages different if one or other path was taken? \citet{charlesworth1985} % E: This is a great question.  We don't answer it with this round of analyses.  Maybe when we have more time for the revision?
% E: Relatedly, I'm wondering if pathways should be a separate paper.  There is the robertson2011 question, the modification that allows for diversification, and the charlesworth1985 question.  And dioecy.
%B: Yes, agreed.

%B: STOPPED HERE. 
Two methodological developments are critical for disentangling the complex interactions between two or more traits and their link to diversification.
First, the model proposed by \citet{fitzjohn_2012} that extended the state-dependent diversification ``*SSE" modeling framework \citep{maddison_2007}, to allow estimating the effect of an arbitrary number of states (enabling multiple states, MuSSE, instead of binary states, BiSSE). 
This allows multiple states to represent trait combinations, particularly useful if one is interested in explicitly testing whether combinations of traits non-additively affect speciation or extinction. 
Second, the model by \citet{beaulieu_2016} HiSSE proposed a first solution to the problem of  elevated false positive type I errors found often in the BiSSE framework \citep{goldberg_2012}. The authors argued that it is unlikely that the variation in speciation and extinction rates is only due to a focal trait but by considering heterogeneity in the diversification process as the null hypothesis, it is possible to parse out the importanceof the focal trait in diversification while other unobserved processes can happen. HiSSE models have effectively shown that some traits associated with diversification in BiSSE analyses are actually not associated with the diversification process, and instead a hidden state is linked to differences in the diversification rate.  Such results raise the question of the identity of the hidden state, or whether it is an approximation of some unknowable heterogeneity, as well as whether interactions between the known and hidden trait were modeled appropriately. 
Third, one can simultaneously consider the effects of multiple known traits using MuSSE, especially in systems where multiple traits are suspected to influence diversification, and where interactions between those traits are well understood. However, MuSSE might also suffer from hight type I errors if there are other processes shaping diversification other than the traits observed \citep{caetano_2018}
Therefore, a fourth option exists: considering both the presence of multiple known traits  linked to the diversification process, but also, some unknown or hidden state changing speciation and extinction rates. 
These four modeling strategies are proposed here in the context of studying complex interactions among traits, linked diversification.

The focus of our study is the individual role of ploidy and breeding systems, as well as their interaction, on the diversification process in Solanaceae. %sans pathway 
Using ploidy and breeding system data from species in Solanaceae, and considering possible heterogeneity in the diversification process, we assess  the influence of these traits on speciation and extinction rates.
First, we present models in which the effect of ploidy and breeding system are considered separately, and compare the inferences to previously published results. 
Second, we consider the addition of a hidden trait to each of the models that examine traits individually, in order to investigate whether the focal trait directly affects differences in diversification, or if such differences are attributable to unobserved but correlated hidden states.
Third, we ask whether the complex interaction between breeding system and ploidy may be explained by the hidden trait states. 
We evaluate a multi-state model to consider both traits linked to diversification, and then consider whether an additional unobserved trait may be driving differences in diversification. In all these models we also investigate the potential effects of diploidization to explore whether the reversion of polyploidy changes diversification interpretations. 
Our results highlight the importance of considering non-additive effects of traits on net diversification rates, especially when there are strong biologically driven correlations amongst them.

%BWe perform model selection in a Bayesian framework to identify which models are fitting better the evidence of the Solanaceae data.
%Considering them jointly, however, reveals that the ploidy connection is removed by incorporating breeding system.
%We further show that the general results are robust to allowing for diploidization and a hidden trait, and something about pathways. % FIXME
%Our results emphasize the importance of considering traits not only in isolation, especially when there are strong correlations between them.
%We investigate their individual and joint effects on diversification, and whether or not adding hidden states at different steps of modeling can help us generate hypothesis about the benefit of searching for a trait that can pass as a hidden state in a simple model. 
%Additionally, a lack of methods that allow for simultaneous inference of effects of other traits, which also influence diversification, can lead to incorrect conclusions about the focal trait \citep{rabosky_2015, beaulieu_2016}.



