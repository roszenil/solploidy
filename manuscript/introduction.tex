\section{Introduction}


% Outline, discussed by Boris and Emma:
%
% we are interested in the evolution of traits and their influence on lineage diversification
%     ref Barrett work
%     any one study focuses on one/few traits
%     but any real organisms have many, so need to enlarge our view
%     mini "here, we..."
% we have tools to try to answer these questions
%     sse models
%     issues
%     but a way forward is to allow for multiple factors that could influence diversification
% ploidy
%     Stebbins
%     Mayrose, Soltis
%     recent paleo WGD, diploidization
% breeding system
%     directionality
%     diversification in Solanaceae
%     correlation with ploidy
%     pathways from SI-D to SC-P: direct or via SC-D
% here, we...
%     diversification: interaction between ploidy and breeding system
%     also: diploidization, pathways


Studying diversification linked to trait evolution\newline
% Here I need help (perhaps will be the last line I write)
%In these context, botanist have been asking how polyploidy shapes patterns of diversification. 
% B: I would start this off with:
%(1) general intro to macroevolutionary inference, 
%(2) intro to two traits that we reasonably perceive as having the potential to influence speciation and extinction rates, and 
%(3) the causal relationship between D/P and SI/SC transitions, which influences their mutual dependence and possibly conflates inference of individual effects on speciation and extinction rates.
The prevalence of polyploidy and its detection across multiple and highly diverse clades of angiosperms inevitably lead to the hypothesis of the importance of polyploidy in the speciation and extinction patterns observed across the flowering plant phylogeny. 
At the same time, a similar question has been asked in the breeding system world, where self-compatibility has evolved multiple times in flowering plants. 
However, the influence in diversification using both polyploidy and self-compatibility information has not been studied simultaneously.

- Why polyploidy and self-compatibility in particular makes for an  interesting study in the context of diversification\newline
In the polyploidy world, an important debate regarding the diversification of angiosperms has been ongoing since the publication of \citet{mayrose_2011}. The authors discovered that using the latest diversification model linked to two states diploid and polyploid, the net diversification of polyploids was much slower than the net diversification rate than polyploids. This result was surprising and an sparked a discussion about the long-term evolutionary consequences of polyploidy. \citet{soltis_2014} questioned if polyploidy should be regarded as an evolutionary dead-end since the net diversification of polyploids was negative, despite overwhelming evidence about the incidence of polyploidy  especially at the root of highly diverse angiosperm claims (ref here). A year later, diversification models were re-tested and corrected, and still found the same pattern \citep{mayrose_2015}, to the disappointment of plenty of botanists there was no denying of the weak trend of diversification that polyploids left behind. The most recent study by \citet{landis_2018} that used not only the presence of polyploidy in the tips but also the number of whole genome duplications in a taxon lineage found that.... \newline  Meanwhile, studies focusing on diversification patterns and  breeding system have consistently found that self-incompatible plants often have higher net diversification rates compare to their self-compatible counterparts (Emma and Boris' papers here, what about papers that are not solanaceae?) 

- Why other traits need to be considered as well\newline


- What other models and studies have done in the past\newline

- What is lacking from past approaches? \newline
There are two key questions that at the time of the polyploidy debate were difficult to ask. The first is if the models used to measure the diversification of polyploids were correct, and the second question is if the models have potentially included more evidence and potential traits that are not polyploidy or other lines of evidence that could be driving the patterns. At the time, in a different context \citet{beaulieu_2016} were finding an alternative solution to the first question, coming up with a new model that could represent the broad heterogeneity of the diversification process and parse out the signal between the trait of interest and the noise in diversification. Their model, the hidden state speciation and extinction model is a key component to detect whether polyploidy our something else unknown but related to it is driving the speciation and extinction patterns that we see in angiosperms. \newline

-What is our proposal to tackle this problem?\newline

- How this paper is structured \newline

The second question (talk here about breeding system and polyploidy, and how this could be different across clades and this is one of the reasons why we focused on Solanaceae).
-Diversification and breeding systems\\
Goldberg and Igic 2012, different  perspectives\\



=
