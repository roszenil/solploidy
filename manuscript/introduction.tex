\section{Introduction}
Linking changes in diversification to polyploidy and breeding system has been one of the key botany questions for the past n number of years. The prevalence of polyploidy, and the multiple evolutionary changes of breeding systems in important angiosperm clades, leads to the inevitable question of the evolutionary changes.

- Diversification and polyploidy\newline
An important debate regarding the diversification of angiosperms has been ongoing since the publication of \citet{mayrose_2011}. The authors discovered that using the latest diversification model linked to two states diploid and polyploid, the net diversification of polyploids was much slower than the net diversification rate than polyploids. This result was surprising and an sparked a discussion about the long-term evolutionary consequences of polyploidy. \citet{soltis_2014} questioned if polyploidy should be regarded as an evolutionary dead-end since the net diversification of polyploids was negative, despite overwhelming evidence about the incidence of polyploidy  especially at the root of highly diverse angiosperm claims (ref here). A year later, diversification models were re-tested and corrected, and still found the same pattern \citep{mayrose_2015}, to the disappointment of plenty of botanists there was no denying of the weak trend of diversification that polyploids left behind. \newline
There are two key questions that at the time of the polyploidy debate were difficult to ask. The first is if the models used to measure the diversification of polyploids were correct, and the second question is if the models have potentially included more evidence and potential traits that are not polyploidy or other lines of evidence that could be driving the patterns. At the time, in a different context \citet{beaulieu_2016} were finding an alternative solution to the first question, coming up with a new model that could represent the broad heterogeneity of the diversification process and parse out the signal between the trait of interest and the noise in diversification. Their model, the hidden state speciation and extinction model is a key component to detect whether polyploidy our something else unknown but related to it is driving the speciation and extinction patterns that we see in angiosperms. \newline

The second question (talk here about breeding system and polyploidy, and how this could be different across clades and this is one of the reasons why we focused on Solanaceae).
-Diversification and breeding systems\\
Goldberg and Igic 2012, different  perspectives\\

-What we did\\
We investigate the relative importance of both b



Debate of Mayrose