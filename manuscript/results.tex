\section{Results}
\subsection{Polyploidy and Diversification Models}
Similarly to the results obtained by \citet{mayrose_2011} and \citet{mayrose_2015}, the D/P polyploidy  model we found that net diversification of diploids is significantly higher than the the net diversification of polyploids. This result holds true whether or not the diploidization parameter is present. However, when diploidization is present the net diversification of polyploids is nonnegative with probability 1, whereas in the absence of diploidization the net diversification of polyploids is negative with probability 0.75(HERE VERIFY QUANTILE). In terms of the relative extinction, when the diploidization parameter is present both polyploids and diploids have posterior distributions that overlap, but that pattern changes in the absence of the diploidization parameter leading to a significant difference between relative extinction where polyploids have a significant higher relative extinction rate. Polyploidization rate $rho$ has a credible interval between (a,b) and diploidization rate has a credible interval between (c,d).

For the D/P- A/B model with diploididization the diploid and polyploid net diversification rates are overlapping for both state A and B of the hidden trait. In this model, the differences in net diversification are due to the presence of a hidden trait and not to the differences in ploidy. When diploidization parameter is absent the pattern holds. Rates of polyploidization have a 95\% credible interval of (a,b) and diploidization rate has a 95\% credible interval ()  for the model with diploidization, and polyploidy rate has a 95\% credible interval (c,d) for the model without diploidization.

\subsection{Breeding System and Diversification models }


\subsection{Bre}



