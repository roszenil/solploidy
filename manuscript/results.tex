\section{Results}
\subsection{Polyploidy and Diversification Models}
Similarly to the results obtained by \citet{mayrose_2011} and \citet{mayrose_2015}, we found that in the D/P polyploidy model the net diversification of diploids is larger than the the net diversification of polyploids since the net diversification distributions do not overlap (Figure 2(A)). This result holds true whether or not the diploidization parameter is present. However in the presence of  the diploidization parameter the net diversification rate of polyploids is nonnegative with probability 1 (Figure 2(A)), whereas in the absence of diploidization the net diversification rate of polyploids can be negative with a probability (HERE verify the quantile) (FIGURE 3(A)). In terms of the relative extinction, when the diploidization parameter is present both polyploids and diploids have posterior distributions that overlap, but that pattern changes in the absence of the diploidization parameter leading to a significant difference between relative extinction where polyploids have a significant higher relative extinction rate (see Supplementary Information).\newline

For the D/P- A/B model with diploididization the diploid and polyploid net diversification rates are overlapping for both state A and B of the hidden trait (Figure 2(B)). In this model, the differences in net diversification are due to the presence of a hidden trait and not to the differences in ploidy. When diploidization parameter is absent the hidden state is still driving the differences in diversification rates (Figure 3(B)).

\subsection{Breeding System and Diversification models }
In the I/C breedyng system model we found that the net diversification rate for self-incompatible state is larger than the net diversification rate for self-compatible state (Figure 2(C)). The net diversification rate for self-compatible has a probability distribution centered at zero.  \newline

When a hidden state is added in the I/C-A/B model, we found that under the hidden state A the self-compatible and self-incompatible net diversification rates are different. Under the hidden state B, those two rates overlap with a probability (HERE CALCULATE THAT) meaning that they are different with probability  (CALCULATE) (Figure 2(D)). These results agree with previous results found by  \citet{goldberg_2012}. However, \citet{goldberg_2012} used a ClaSSE  approach since they were interested in anagenetic and cladogenetic changes for self-incompatibility using a smaller subset of the data presented in the current work.

% Describe rate q_IC and hidden state rate

\subsection{Polyploidy and Breeding Sytem models}
In the ID/P/CD model we found that self-incompatible and diploid state has a significantly larger net diversification rate compare to both self-compatible diploid and polyploid rates. Meanwhile, both self-compatible diploid and polyploid posterior distributions of net diversification rates completely overlap (Figure 2(E)). When hidden state was added in the ID/P/CD-A/B model, we observed significant differences between self-compatible and self-incompatible diploids for both A and B values of the hidden state. Self-incompatible state had a larger net diversification rate than self compatible for both A and B states. However, the posterior distribution for the net diversification rate of polyploids overlaps with both the self-compatible and self-incompatible posterior distributions for each value of the hidden state(Figure 2F) meaning that polyploidy state is not significantly different from diploid in net diversification terms.  The resulting effect of adding the hidden state values is significant 


\subsection{Diploidization as an exploratory hypothesis}
In the D/P model  the diploidization rate $\delta$ and polyploidization rate $\rho$ are different from zero with probability 1. Diploidization rate is more uncertain than polyploidization. For the D/P no $\delta$ model, the polyploidization rate is still different from zero but has a wider 95\% credible interval (see Supplementary Information). In the D/P-A/B model the diploidization rate remains positive but there rate of polyploidization becomes really uncertain. Whereas, in the absence of diploidzation (D/P no $\delta$ A/B) the rate of polyploidy has a small credible interval \newline

For the model ID/P/CD containing both polyploidy and breeding system traits, we found that the diploidization rate is really uncertain with a MAP that is (CALCULATE VALUE HERE- close to zero). The polyploidization rate from self-incompatible diploid $\rho_I$ is slightly faster than the polyploidization rate from self-compatible diploid $\rho_C$  and that pattern remains the same for the ID/P no $\delta$/CD model(see supplementary information).
%Here I'm missing the MuHiSSE plots.

\subsection{Model selection}

In \cref{table:marginallike} we list the marginal likelihood in log scale for each of the models tested. In the table we show what are the different components included and excluded for each model as a a summary of the diagrams from \cref{figure:netdivall}. From the marginal likelihoods in log scale, the Bayes factors in log-scale were calculated as shown in table \cref{table:bayesfactors}. After testing every single pair of polyploidy modes (1-4) we found overwhelming evidence that the best polyploidy model is always  the D/P-A/B, that is the model with hidden state and diploidization. \newline
For the two models following the evolution of breeding system, the I/C-A/B is the best choice between models 3 and 4.\newline
The models that follow the diversification linked to both polyploidy and breeding system are the last 4 (models 5-8). When comparing using Bayes factores every two models we found that the IC/P/CD-A/B is always preferred over the rest, meaning that the model that has a hidden state and diploidization is chosen over the ones that lack either of both of those options.\newline
Therefore, the models that were chosen were always the ones containing a hidden trait, and in the case of polyploidy models, the ones containing a diploidization parameter $\delta$ are preferable.



Example in \cref{suppfigure:example}. % XXX

\begin{figure}
\includegraphics[width=\textwidth]{Netdiversificationallmodels.pdf}
  \caption{Net diversification rates for all models that include diploidization.}  
\label{figure:netdivall}
\end{figure}

\begin{suppfigure}
\caption{example Supp Info figure} % XXX
\label{suppfigure:example}
\end{suppfigure}
