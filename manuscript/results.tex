\section{Results}

\subsection{Models}
\subsubsection{Ploidy and diversification}
When considering ploidy alone (M1. $D/P$ model), we found a larger net diversification rate for diploids than for polyploids, in agreement with \citet{mayrose_2011, mayrose_2015}.
This result holds with (\cref{figure:netdivall}A) or with the diploidization parameter (model M6 \cref{suppfigure:alldip}A) .
Incorporating a hidden state in this model, however, reduces the clear separation in diversification rate estimates between diploids and polyploids (M4. $D/P+A/B$ asym model; \cref{figure:netdivall}B, and M9 \cref{suppfigure:alldip}B).
Thus, differences in net diversification rates are explained by an unknown factor.
The lack of differences between diploid and polyploid net diversification rates is also supported by the Bayes factors shown in table \cref{table:bayesfactors} where both model M4 and the character independent model M2 have equal evidence of  being preferred over every other model of ploidy. See table \cref{supptable:M6M10} showing analogous results when diploidzation is accounted for.

\subsubsection{Breeding system and diversification}
When considering breeding system alone (M11. $I/C$ model, \cref{figure:netdivall}C), we found a larger net diversification rate for SI than for SC species, in agreement with \citet{goldberg_2010}.
When a hidden state is included (M14. $I/C+A/B$ model), the large net diversification rate difference persists for hidden state $A$ but and it slightly overlaps in the  hidden state $B$ (\cref{figure:netdivall}D).
When calculating Bayes factors for all breeding system models (M11-M15) we found that the best two supported models are the models with at least asymmetric hidden states (M14 and M15 \cref{table:M11M15})
Thus, differences in net diversification rates are best explained by both breeding system and an unknown factor \citep[as in Onagraceae;][]{freyman_2019} and not only by the focal trait nor an unknown factor.
% The transition rate from SI to SC is $q_{IC}=0.3$ for both the I/C and the I/C+A/B models.

\subsubsection{Ploidy, breeding system, and diversification}
When considering ploidy and breeding system together (M16. $ID/CD/CP$ model), the net diversification rate for SI diploids was greater than for either SC diploids or SC polyploids, with or without diploidization (M16 \cref{figure:netdivall}E, M21\cref{suppfigure:alldip}E).
Thus, the difference in net diversification associated with breeding system persists even when ploidy is included in the model.
The reverse is not true, the association of ploidy with net diversification in the M1. $D/P$ model (\cref{figure:netdivall}A, \cref{suppfigure:alldip}A) appears to be driven by the subset of diploids that are SI; among SC species, net diversification rates for diploids and polyploids is not distinct.

When a hidden state is included (M19. $IC/CD/CP+A/B$ model) the net diversification rate separation amongst states ID and (CD, CP)  is clear under hidden state A, but under hidden state B the separation exist between states ID and CP, while CD overlaps with the other two states (\cref{figure:netdivall}F).
The same general pattern remains when diploidization is included (\cref{suppfigure:alldip}F), but the hidden states are reversed in the x-axis.
Bayes factors for the three state models that include the character independent model show very strong support preferring asymmetric hidden state with the focal traits models shown in  \cref{supptable:M16M20} (with diploidization \cref{supptable:M21M25}).

\subsubsection{Lumped models}
When comparing model M26. Lumped $D/P$ against M16. $ID/CD/CP$ (\cref{figure:lumped}B and C) Bayes factors reveal moderate support ($K=4.1$) preferring the three state model instead of the two state model (\cref{table:lumped}).
A similar result was obtained when comparing the lumped model for ploidy but in the presence of hidden states (model M27 \textit{vs.} M23 \cref{table:lumped}, \cref{suppfigure:lumpedDP}E and F).

We also compared lumped models for breeding system to test for the inclusion of ploidy state (M28 \cref{figure:lumped}E \textit{vs.} M16 \cref{figure:lumped}C) and we found no evidence preferring the three-state over the two state-model ($K=-0.6$, \cref{table:lumped}).
However, when accounting for hidden state,  the three-state model M18. $I/C/CD/CP+A/B$ (\cref{suppfigure:lumpedIC}F) was moderately preferred ($K=2.6$) over the lumped model M29.(\cref{suppfigure:lumpedIC}E).

\subsection{Pathways to polyploidy}
There are two pathways by which SI diploid lineages eventually---given enough time---become SC polyploids.
In the one-step pathway, polyploidization directly disables SI.
In the two-step pathway, SI is first lost within the diploid state, followed by polyploidization.
Determining the relative contribution of these pathways based on our estimated transition rates from the $ID/CD/CP$ model (median rate values from M16), we find that the one-step pathway is more likely on short timescales and the two-step pathway is more likely on long timescales (\cref{figure:pathways}, left panels).
Beginning with a single SI diploid lineage, when not much time has elapsed, the one-step pathway is more likely because it only necessitates a single event to effect transition to the SC polyploid state.
When more time has elapsed, the two-step pathway is more likely because the rate of loss of SI within diploids, $q_{IC}$, is greater than the rate of polyploidization for SI species, $\rho_I$ (\cref{suppfigure:IDCDCPnodip}).
That is, an $ID$ lineage is more likely to begin its path to polyploidy with a transition to $CD$, but completing this path to $CP$ takes longer.
(The results are qualitatively the same when using transition rate estimates from the model that does not allow diversification differences related to the observed states, M17).

The preceding conclusions, however, ignore the changes in numbers of lineages in each state due to speciation and extinction.
By analogy, envisioning the states as stepping stones, the extent to which each stone grows or shrinks over time affects the utility of each possible path.
Allowing for the different net diversification rates for each state (again using median rate estimates from M16), we find a qualitative difference in the relative pathway contributions.
In particular, the lower rate of net diversification in the $CD$ state, relative to $ID$, means that relatively fewer lineages are available to complete the second step of the two-step pathway, ending in $CP$.
Consequently, even over long timescales, the two-step pathway contributes less to the formation of polyploids (\cref{figure:pathways}, right panels).

\subsection{Model selection key questions}
\subsubsection{Are diversification patterns only determined by hidden states?}
Whether the determination of the diversification is dependent on the focal traits and not only on the hidden state was done via the comparisons of character independent models against the focal traits w/o hidden states models (results shown in  \cref{table:bayesfactors,table:M11M15,supptable:M6M10,supptable:M16M20,supptable:M21M25}).
The general pattern is to moderate- to strongly prefer models with focal traits and asymmetric hidden states over the rest of the models with the exception of the ploidy only model where the character independent model (M2) seems to be equally preferred($K=0.5$) as the model with ploidy and asymmetric hidden rates (M4, \cref{table:bayesfactors}).

\subsubsection{Are hidden states necessary?}
In the comparisons of simple diversification models of focal traits against models with hidden states we found that models with asymmetric hidden traits are strongly preferred over simpler models (summarized in table \cref{supptable:testaddhidden}).

\subsubsection{Is a second trait adding information to the diversification process?}
As summarized in  \cref{table:lumped}, it is moderately preferred to add breeding system to ploidy only models (\cref{suppfigure:lumpedDP}).
When adding ploidy information to breeding system models to create a three-state model, we found that there is no evidence that it is preferred (M28 vs. M16 \cref{suppfigure:lumpedIC}B and C). 
However, it becomes moderately preferable in the presence of hidden states (M29 vs. M23  \cref{suppfigure:lumpedIC}E and C).

\subsubsection{Is there evidence for diploidization?}
We considered models both with and without diploidization in order to explore its effects on the estimates of state-dependent diversification.
For the models that only include diploid and polyploid states (M6-M10) the results are similar in terms of the effect of diversification (\cref{suppfigure:alldip}).
When comparing models with diploidization against models without diploidization we found moderate evidence that models containing the parameter $\delta$ are preferred over models without diploidization (summarized in \cref{supptable:testdiploidization}).

\subsubsection{Do asymmetric rates in hidden states models affect the pattern of diversification?}
When comparing different types of asymmetry across hidden states models we found that the asymmetry did not change the direction of effects in net diversification results . However, models with asymmetric hidden rates reduced the overlap amongst net diversification posterior distributions as shown in \cref{suppfigure:asymmetric}. Furthermore, hidden state models with asymmetric traits are strongly preferred over models with equal rates between hidden states (summarized in \cref{supptable:asymmetry}).

