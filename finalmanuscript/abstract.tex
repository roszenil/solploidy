\section{Summary}

% presently 187/200 words
\begin{itemize}

% Rationale
\item If particular traits consistently affect rates of speciation and extinction, broad macroevolutionary patterns can be interpreted as consequences of selection at high levels of the biological hierarchy.
Identifying traits associated with diversification rates is difficult because of the wide variety of characters under consideration and the statistical challenges of testing for associations from comparative phylogenetic data.
Ploidy (diploid \vs polyploid states) and breeding system (self-incompatible \vs self-compatible states) are both thought to be drivers of differential diversification in angiosperms.

% Methods
\item  We fit twenty-nine diversification models to extensive trait and phylogenetic data in Solanaceae and investigate how speciation and extinction rate differences are associated with ploidy, breeding system, and the interaction between these traits.%B: I re-ordered this and removed the second "to": To extensive trait and phylogenetic data in Solanaceae, we fit twenty-nine diversification models to investigate how speciation and extinction rate differences are associated with ploidy, breeding system, and the interaction between these traits.

% Results
\item We show that diversification patterns in Solanaceae are better explained by breeding system and an additional unobserved factor, rather than by ploidy. 
We also find that the most common evolutionary pathway to polyploidy in Solanaceae occurs via direct breakdown of self-incompatibility by whole-genome duplication, rather than indirectly via breakdown followed by polyploidization.

% Conclusions
\item Comparing multiple stochastic diversification models that include complex trait interactions alongside hidden states enhances our understanding of the macroevolutionary patterns in plant phylogenies. 

\end{itemize}
